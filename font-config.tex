% Einfach mittels % Einfach mittels % Einfach mittels % Einfach mittels \input{font-config} einbinden.

\usepackage{ifxetex}

\ifxetex
  \def\xelatexmainfont{Minion Pro}
  \def\xelatexmainboldfont{Minion Pro Bold}
  \def\xelatexsansfont{Gill Sans Light}
  \def\xelatexsansboldfont{Gill Sans}
  \def\xelatextitlepagefont{Gill Sans}
  \def\xelatexmonofont{Fira Mono}
  \def\xelatexmathfont{Minion Pro}

  % Um auch schöne Schriftarten auswählen zu können:
  \usepackage[MnSymbol]{mathspec}

  % Wir wollen, dass alle unsere Schriften für TeX und einander angepasst sind:
  \defaultfontfeatures{
    Ligatures=TeX,
    Scale=MatchLowercase,
  }

  % Die Hauptschriftart:
  \setmainfont{\xelatexmainfont}[
    Mapping=TeX-text,
    BoldFont={\xelatexmainboldfont},
  ]
  % Die Matheschriftart:
  \setmathfont(Digits,Latin,Greek)[
    Numbers={Lining, Proportional}
  ]{\xelatexmathfont}
  \setmathrm{\xelatexmathfont}
  % Die Schriftart für serifenlose Texte (z.B. Überschriften):
  \setallsansfonts[
    Mapping=TeX-text,
    BoldFont={\xelatexsansboldfont},
  ]{\xelatexsansfont}
  % Und die Schriftart für nichtproportionale Texte:
  \setallmonofonts[Scale=MatchLowercase]{\xelatexmonofont}
  \newfontface\titlepagefont{\xelatextitlepagefont}
\else
  \usepackage[utf8]{inputenc}
\fi
 einbinden.

\usepackage{ifxetex}

\ifxetex
  \def\xelatexmainfont{Minion Pro}
  \def\xelatexmainboldfont{Minion Pro Bold}
  \def\xelatexsansfont{Gill Sans Light}
  \def\xelatexsansboldfont{Gill Sans}
  \def\xelatextitlepagefont{Gill Sans}
  \def\xelatexmonofont{Fira Mono}
  \def\xelatexmathfont{Minion Pro}

  % Um auch schöne Schriftarten auswählen zu können:
  \usepackage[MnSymbol]{mathspec}

  % Wir wollen, dass alle unsere Schriften für TeX und einander angepasst sind:
  \defaultfontfeatures{
    Ligatures=TeX,
    Scale=MatchLowercase,
  }

  % Die Hauptschriftart:
  \setmainfont{\xelatexmainfont}[
    Mapping=TeX-text,
    BoldFont={\xelatexmainboldfont},
  ]
  % Die Matheschriftart:
  \setmathfont(Digits,Latin,Greek)[
    Numbers={Lining, Proportional}
  ]{\xelatexmathfont}
  \setmathrm{\xelatexmathfont}
  % Die Schriftart für serifenlose Texte (z.B. Überschriften):
  \setallsansfonts[
    Mapping=TeX-text,
    BoldFont={\xelatexsansboldfont},
  ]{\xelatexsansfont}
  % Und die Schriftart für nichtproportionale Texte:
  \setallmonofonts[Scale=MatchLowercase]{\xelatexmonofont}
  \newfontface\titlepagefont{\xelatextitlepagefont}
\else
  \usepackage[utf8]{inputenc}
\fi
 einbinden.

\usepackage{ifxetex}

\ifxetex
  \def\xelatexmainfont{Minion Pro}
  \def\xelatexmainboldfont{Minion Pro Bold}
  \def\xelatexsansfont{Gill Sans Light}
  \def\xelatexsansboldfont{Gill Sans}
  \def\xelatextitlepagefont{Gill Sans}
  \def\xelatexmonofont{Fira Mono}
  \def\xelatexmathfont{Minion Pro}

  % Um auch schöne Schriftarten auswählen zu können:
  \usepackage[MnSymbol]{mathspec}

  % Wir wollen, dass alle unsere Schriften für TeX und einander angepasst sind:
  \defaultfontfeatures{
    Ligatures=TeX,
    Scale=MatchLowercase,
  }

  % Die Hauptschriftart:
  \setmainfont{\xelatexmainfont}[
    Mapping=TeX-text,
    BoldFont={\xelatexmainboldfont},
  ]
  % Die Matheschriftart:
  \setmathfont(Digits,Latin,Greek)[
    Numbers={Lining, Proportional}
  ]{\xelatexmathfont}
  \setmathrm{\xelatexmathfont}
  % Die Schriftart für serifenlose Texte (z.B. Überschriften):
  \setallsansfonts[
    Mapping=TeX-text,
    BoldFont={\xelatexsansboldfont},
  ]{\xelatexsansfont}
  % Und die Schriftart für nichtproportionale Texte:
  \setallmonofonts[Scale=MatchLowercase]{\xelatexmonofont}
  \newfontface\titlepagefont{\xelatextitlepagefont}
\else
  \usepackage[utf8]{inputenc}
\fi
 einbinden.

\usepackage{ifxetex}

\ifxetex
  \def\xelatexmainfont{Minion Pro}
  \def\xelatexmainboldfont{Minion Pro Bold}
  \def\xelatexsansfont{Gill Sans Light}
  \def\xelatexsansboldfont{Gill Sans}
  \def\xelatextitlepagefont{Gill Sans}
  \def\xelatexmonofont{Fira Mono}
  \def\xelatexmathfont{Minion Pro}

  % Um auch schöne Schriftarten auswählen zu können:
  \usepackage[MnSymbol]{mathspec}

  % Wir wollen, dass alle unsere Schriften für TeX und einander angepasst sind:
  \defaultfontfeatures{
    Ligatures=TeX,
    Scale=MatchLowercase,
  }

  % Die Hauptschriftart:
  \setmainfont{\xelatexmainfont}[
    Mapping=TeX-text,
    BoldFont={\xelatexmainboldfont},
  ]
  % Die Matheschriftart:
  \setmathfont(Digits,Latin,Greek)[
    Numbers={Lining, Proportional}
  ]{\xelatexmathfont}
  \setmathrm{\xelatexmathfont}
  % Die Schriftart für serifenlose Texte (z.B. Überschriften):
  \setallsansfonts[
    Mapping=TeX-text,
    BoldFont={\xelatexsansboldfont},
  ]{\xelatexsansfont}
  % Und die Schriftart für nichtproportionale Texte:
  \setallmonofonts[Scale=MatchLowercase]{\xelatexmonofont}
  \newfontface\titlepagefont{\xelatextitlepagefont}
\else
  \usepackage[utf8]{inputenc}
\fi
