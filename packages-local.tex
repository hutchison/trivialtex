% Über die Jahre haben sich einige Pakete als nützlich erwiesen.
% Damit ich nicht bei jedem neuen Dokument raussuchen muss, welches Paket wofür gebraucht
% wird, gibt’s hier die Zusammenfassung.

% Einfach mittels % Über die Jahre haben sich einige Pakete als nützlich erwiesen.
% Damit ich nicht bei jedem neuen Dokument raussuchen muss, welches Paket wofür gebraucht
% wird, gibt’s hier die Zusammenfassung.

% Einfach mittels % Über die Jahre haben sich einige Pakete als nützlich erwiesen.
% Damit ich nicht bei jedem neuen Dokument raussuchen muss, welches Paket wofür gebraucht
% wird, gibt’s hier die Zusammenfassung.

% Einfach mittels % Über die Jahre haben sich einige Pakete als nützlich erwiesen.
% Damit ich nicht bei jedem neuen Dokument raussuchen muss, welches Paket wofür gebraucht
% wird, gibt’s hier die Zusammenfassung.

% Einfach mittels \input{packages-local} einbinden.
% Wenn die Dokumentenklasse tufte-handout oder tufte-book ist, dann wird das hier
% automatisch geladen.

\makeatletter
\@ifclassloaded{tufte-handout}{%
  % Bei Tufte sind die Schriften schon eingestellt.
  \typeout{Current class is tufte-handout, fonts are already configured.}
}{%
  \usepackage{ifxetex}
  \ifxetex
    \def\xelatexmainfont{Minion Pro}
    \def\xelatexmainboldfont{Minion Pro Bold}
    \def\xelatexsansfont{Gill Sans Light}
    \def\xelatexsansboldfont{Gill Sans}
    \def\xelatextitlepagefont{Gill Sans}
    \def\xelatexmonofont{Fira Mono}
    \def\xelatexmathfont{Minion Pro}

    % Deutsche Sprache bei Silbentrennung und Datum:
    \usepackage{polyglossia}
    \setdefaultlanguage[babelshorthands=true]{german}

    % Um auch schöne Schriftarten auswählen zu können:
    \usepackage[MnSymbol]{mathspec}

    % Wir wollen, dass alle unsere Schriften für TeX und einander angepasst sind:
    \defaultfontfeatures{
      Ligatures=TeX,
      Mapping=TeX-text,
      Scale=MatchLowercase,
    }

    % Die Hauptschriftart:
    \setmainfont{\xelatexmainfont}[
      BoldFont={\xelatexmainboldfont},
    ]
    % Die Matheschriftart:
    \setmathfont(Digits,Latin,Greek)[
      Numbers={Lining, Proportional}
    ]{\xelatexmathfont}
    \setmathrm{\xelatexmathfont}
    % Die Schriftart für serifenlose Texte (z.B. Überschriften):
    \setallsansfonts[
      BoldFont={\xelatexsansboldfont},
    ]{\xelatexsansfont}
    % Und die Schriftart für nichtproportionale Texte:
    \setallmonofonts[]{\xelatexmonofont}
    \newfontface\titlepagefont{\xelatextitlepagefont}
  \else
    \usepackage[utf8]{inputenc}

    % Deutsche Sprache bei Silbentrennung und Datum:
    \usepackage[ngerman]{babel}
  \fi
}
\makeatother

% _das_ Mathepaket schlechthin:
%\usepackage[
%  %% Nummerierung von Gleichungen links:
%  leqno,
%  %% Ausgabe von Gleichungen linksbündig:
%  fleqn,
%]{mathtools}
% und dazu noch ein paar Mathesymbole und so:
% (muss vor mathspec geladen werden)
%\usepackage{amsmath, amssymb}

%\usepackage{microtype}

% https://ctan.org/pkg/parskip
%\usepackage[
%%skip=0.5baselineskip plus 2pt,  % Der Abstand zwischen zwei Absätzen.
%%indent=0pt, % Die Einrückung der ersten Zeile eines Absatzes.
%%parfill=30pt, % Der Abstand des Endes der letzten Zeile eines Absatzes und dem rechten Seitenrand.
%]{parskip}

% chemische Formeln
% https://ctan.org/pkg/mhchem
%\usepackage{mhchem}


% St. Mary Road, liefert Symbole für theoretische Informatik (z.B. \lightning):
% https://www.ctan.org/pkg/stmaryrd
%\usepackage{stmaryrd}

% nutze die volle Seite als Satzspiegel:
% https://www.ctan.org/pkg/fullpage
%\usepackage[
  % Randbreite sei 1.5cm (sonst ist sie 1in):
  %cm,
  % Kopf- und Fußzeile werden miteinbezogen:
  %headings
%]{fullpage}

% verbesserte Tabellen
% https://www.ctan.org/pkg/array
% bietet u.a. die Spaltenmöglichkeit 'm{width}' = zentrierte Spalte mit fester Breite
%\usepackage{array}

% kann komplexe Linien in Tabellen ziehen:
% https://www.ctan.org/pkg/hhline
%\usepackage{hhline}

% mehrseitige Tabellen:
% https://www.ctan.org/pkg/longtable
%\usepackage{longtable}

% Tabellen mit gedehnten Spalten:
% https://www.ctan.org/pkg/tabularx
%\usepackage{tabularx}
% vielleicht auch tabulary anschauen:
% https://www.ctan.org/pkg/tabulary
%\usepackage{tabulary}

% Pimpt enumerate auf (optionales Argument liefert Nummerierung):
\usepackage{enumerate}

% Kann descriptions auf die selbe Höhe bringen:
% https://www.ctan.org/pkg/enumitem
%\usepackage{enumitem}

% Liefert Hyperlinks (\hyperref, \url, \href}
\usepackage{hyperref}
%\hypersetup{%
%  colorlinks=false,
%  linkcolor=black,
%  urlcolor=blue,
%}

% To automatically typeset a cross-reference according to the type of thing referred to,
% simply refer to it using \cref{<label>}.
% https://www.ctan.org/pkg/cleveref
%\usepackage{cleveref}

% Farben (wie bei TikZ):
%\usepackage[dvipsnames]{xcolor}
%\definecolor{mygray}{gray}{0.8}

% Access to PostScript standard Symbol and Dingbats fonts
% https://www.ctan.org/pkg/pifont
%\usepackage[]{pifont}

% Ändert den Zeilenabstand:
%\usepackage[
%  % nur eine Möglichkeit auswählen:
%  singlespacing
%  %onehalfspacing
%  %doublespacing
%]{setspace}

% Schönere Tabellen
% dazu gibt's neue Kommandos:
% - \toprule[(Dicke)], \midrule[(Dicke)], \bottomrule[(Dicke)]
% - \addlinespace: Extrahöhe zwischen Zeilen
\usepackage{booktabs}

% TODOs:
%\usepackage[
%  ngerman,
%  textwidth=2cm,
%  textsize=tiny,
%  backgroundcolor=white,
%  linecolor=black,
%]{todonotes}

% Schöne numerische Zitierungen:
%\usepackage{cite}
%\usepackage[square, numbers]{natbib}

% Ermöglicht durch \begin{linenumbers} Zeilennummern anzuzeigen:
%\usepackage{lineno}

% Ermöglicht Zugriff auf die letzte Seite (z.B. \pageref{LastPage}):
\usepackage{lastpage}

% Logische Beweise:
%\usepackage{bussproofs}

% Unterstreichungen (\uline, \uuline, \sout: durchgestrichen, \uwave):
%\usepackage{ulem}

% Kann alle möglichen Maße ändern
% will man Querformat, dann:
%\usepackage[landscape]{geometry}

% bietet gestrichelte vert. Linien in Tabellen (':')
%\usepackage{arydshln}

% Quelltext schön setzen:
%\usepackage{listings}
% Quelltext noch schöner setzen:
% (verlangt "xelatex -shell-escape"!)
%\usepackage{minted}

% Algorithmen und Pseudocode:
%\usepackage{algorithm}
%\usepackage{algorithmic}
%\floatname{algorithm}{Algorithmus}
%\renewcommand{\algorithmicrequire}{\textbf{Eingabe:}}
%\renewcommand{\algorithmicensure}{\textbf{Ausgabe:}}

% Bilder einbinden:
%\usepackage{graphicx}
%\usepackage{subcaption}

% Verbessert den Satz von Abbildungsüberschriften:
%\usepackage{caption}

% um in Tabellen eine Zelle über mehrere Zeilen laufen zu lassen:
%\usepackage{multirow}

%\usepackage{float}

% SI-Einheiten mittels \si{}:
%\usepackage[mode=text]{siunitx}
%\sisetup{%
%  output-decimal-marker={,},
%  per-mode=fraction,
%  exponent-product=\cdot,
%}
%\DeclareSIUnit\cal{cal}
%\DeclareSIUnit\diopter{dpt}
%\DeclareSIUnit\fahrenheit{F}
%\DeclareSIUnit\molar{\textsc{m}}
%\DeclareSIUnit\pH{pH}
%\DeclareSIUnit\gewprozent{Gew\%}
%\DeclareSIUnit\poise{P}

% nette Brüche mittels \sfrac{}{}:
%\usepackage{xfrac}

% Coole Zeichnungen:
%\usepackage{tikz}
%\usetikzlibrary{%
  %backgrounds,
  %mindmap,
  %shapes.geometric,
  %shapes.symbols,
  %shapes.misc,
  %shapes.multipart,
  %positioning,
  %fit,
  %calc,
  %arrows,
  %automata,
  %trees,
  %decorations.pathreplacing,
  %circuits.ee.IEC,
  %intersections,
  %through,
%}

%\usepackage{pgfplots}
%\pgfplotsset{compat=1.16}

% eigens gebaute Lochmarken:
%\usepackage{eso-pic}
%\AddToShipoutPicture*{
  %\put(\LenToUnit{0mm},\LenToUnit{228.5mm})
    %{\rule{\LenToUnit{20pt}}{\LenToUnit{0.5pt}}}
  %\put(\LenToUnit{0mm},\LenToUnit{68.5mm})
    %{\rule{\LenToUnit{20pt}}{\LenToUnit{0.5pt}}}
%}

%\usepackage{titlesec}
%\titleformat*{\paragraph}{\itshape\mdseries} % chktex 6
% \titleformat{\section}
%   {\sffamily}{\thesection}{1em}{}

% ein Eintrag in einer description-Liste wird in ganz normaler Schrift angezeigt (kein
% sans-serif, kein fett):
%\renewcommand{\descriptionlabel}[1]{\hspace{\labelsep}#1}

%\usepackage{abstract}
%\addto\captionsngerman{\renewcommand{\abstractname}{Abstract}}

%\usepackage{epigraph}
%\setlength{\epigraphwidth}{0.42\textwidth}

% Definitionen und Sätze:
%\usepackage[]{amsthm}
% Definitionen, Probleme werden alle mit einem Zeilenumbruch gesetzt, z.B.:
%   Definition 1 (Graph)
%   Ein Graph G = (V, E) ist …
%\newtheoremstyle{bonny}% schottisch für „ansehnlich“
%  {9pt}% measure of space to leave above the theorem. E.g.: 3pt
%  {6pt}% measure of space to leave below the theorem. E.g.: 3pt
%  {}% name of font to use in the body of the theorem
%  {}% measure of space to indent
%  {\bfseries}% name of head font
%  {\smallskip}% punctuation between head and body
%  {\newline}% space after theorem head; " " = normal interword space
%  {}% Manually specify head

%\theoremstyle{bonny}
%\newtheorem{definition}{Definition}
%\newtheorem{problem}{Problem}

% Beispiele, Sätze, Theoreme, Lemmata werden alle *ohne* Zeilenumbruch gesetzt, z.B.:
%   Satz 1  Für alle Graphen gilt …
%\newtheoremstyle{sweet}%
%  {9pt}% measure of space to leave above the theorem. E.g.: 3pt
%  {6pt}% measure of space to leave below the theorem. E.g.: 3pt
%  {}% name of font to use in the body of the theorem
%  {}% measure of space to indent
%  {\bfseries}% name of head font
%  {}% punctuation between head and body
%  {1em}% space after theorem head; " " = normal interword space
%  {}% Manually specify head

%\theoremstyle{sweet}
%\newtheorem{beispiel}{Beispiel}
%\newtheorem{satz}{Satz}
%\newtheorem{theorem}{Theorem}
%\newtheorem{lemma}{Lemma}
%\newtheorem{folgerung}{Folgerung}

%% coole Kopf- und Fußzeilen:
%\usepackage{fancyhdr}
%% Seitenstil ist natürlich fancy:
%\pagestyle{fancy}
%% alle Felder löschen:
%\fancyhf{}
%% Veranstaltung:
%%\fancyhead[L]{}
%% Seriennummer:
%%\fancyhead[C]{}
%% Name und Matrikelnummer:
%%\fancyhead[R]{}
%%\fancyfoot[L]{}
%\fancyfoot[C]{\thepage}
%%\fancyfoot[C]{\thepage\,/\,\pageref{LastPage}}
%% Linie oben/unten:
%\renewcommand{\headrulewidth}{0.0pt}
%\renewcommand{\footrulewidth}{0.0pt}

%\newcommand{\cmark}{\ding{51}}%
%\newcommand{\xmark}{\ding{55}}%
%\newcommand{\richtig}{\textcolor{ForestGreen}{\cmark}}
%\newcommand{\falsch}{\textcolor{BrickRed}{\xmark}}

%\newcommand{\unterschrift}[2][5cm]{%
%  \begin{tabular}{@{}p{#1}@{}}
%    #2 \\[2\normalbaselineskip]
%    \hrule \\[-12pt]
%    {\small Unterschrift} \\[2\normalbaselineskip]
%    \hrule \\[-12pt]
%    {\small Datum}
%  \end{tabular}
%}

%\newcommand{\BigO}{\mathcal{O}}

%\newenvironment{notiz}{
%  \color{Maroon}
%  \paragraph*{Notiz}
%}{
%  \color{black}
%}

%\newcommand{\p}[1]{\text{p#1}}
%\newcommand{\pKs}{\text{pK$_\text{S}$}}
%\newcommand{\pKb}{\text{pK$_\text{B}$}}
 einbinden.
% Wenn die Dokumentenklasse tufte-handout oder tufte-book ist, dann wird das hier
% automatisch geladen.

\makeatletter
\@ifclassloaded{tufte-handout}{%
  % Bei Tufte sind die Schriften schon eingestellt.
  \typeout{Current class is tufte-handout, fonts are already configured.}
}{%
  \usepackage{ifxetex}
  \ifxetex
    \def\xelatexmainfont{Minion Pro}
    \def\xelatexmainboldfont{Minion Pro Bold}
    \def\xelatexsansfont{Gill Sans Light}
    \def\xelatexsansboldfont{Gill Sans}
    \def\xelatextitlepagefont{Gill Sans}
    \def\xelatexmonofont{Fira Mono}
    \def\xelatexmathfont{Minion Pro}

    % Deutsche Sprache bei Silbentrennung und Datum:
    \usepackage{polyglossia}
    \setdefaultlanguage[babelshorthands=true]{german}

    % Um auch schöne Schriftarten auswählen zu können:
    \usepackage[MnSymbol]{mathspec}

    % Wir wollen, dass alle unsere Schriften für TeX und einander angepasst sind:
    \defaultfontfeatures{
      Ligatures=TeX,
      Mapping=TeX-text,
      Scale=MatchLowercase,
    }

    % Die Hauptschriftart:
    \setmainfont{\xelatexmainfont}[
      BoldFont={\xelatexmainboldfont},
    ]
    % Die Matheschriftart:
    \setmathfont(Digits,Latin,Greek)[
      Numbers={Lining, Proportional}
    ]{\xelatexmathfont}
    \setmathrm{\xelatexmathfont}
    % Die Schriftart für serifenlose Texte (z.B. Überschriften):
    \setallsansfonts[
      BoldFont={\xelatexsansboldfont},
    ]{\xelatexsansfont}
    % Und die Schriftart für nichtproportionale Texte:
    \setallmonofonts[]{\xelatexmonofont}
    \newfontface\titlepagefont{\xelatextitlepagefont}
  \else
    \usepackage[utf8]{inputenc}

    % Deutsche Sprache bei Silbentrennung und Datum:
    \usepackage[ngerman]{babel}
  \fi
}
\makeatother

% _das_ Mathepaket schlechthin:
%\usepackage[
%  %% Nummerierung von Gleichungen links:
%  leqno,
%  %% Ausgabe von Gleichungen linksbündig:
%  fleqn,
%]{mathtools}
% und dazu noch ein paar Mathesymbole und so:
% (muss vor mathspec geladen werden)
%\usepackage{amsmath, amssymb}

%\usepackage{microtype}

% https://ctan.org/pkg/parskip
%\usepackage[
%%skip=0.5baselineskip plus 2pt,  % Der Abstand zwischen zwei Absätzen.
%%indent=0pt, % Die Einrückung der ersten Zeile eines Absatzes.
%%parfill=30pt, % Der Abstand des Endes der letzten Zeile eines Absatzes und dem rechten Seitenrand.
%]{parskip}

% chemische Formeln
% https://ctan.org/pkg/mhchem
%\usepackage{mhchem}


% St. Mary Road, liefert Symbole für theoretische Informatik (z.B. \lightning):
% https://www.ctan.org/pkg/stmaryrd
%\usepackage{stmaryrd}

% nutze die volle Seite als Satzspiegel:
% https://www.ctan.org/pkg/fullpage
%\usepackage[
  % Randbreite sei 1.5cm (sonst ist sie 1in):
  %cm,
  % Kopf- und Fußzeile werden miteinbezogen:
  %headings
%]{fullpage}

% verbesserte Tabellen
% https://www.ctan.org/pkg/array
% bietet u.a. die Spaltenmöglichkeit 'm{width}' = zentrierte Spalte mit fester Breite
%\usepackage{array}

% kann komplexe Linien in Tabellen ziehen:
% https://www.ctan.org/pkg/hhline
%\usepackage{hhline}

% mehrseitige Tabellen:
% https://www.ctan.org/pkg/longtable
%\usepackage{longtable}

% Tabellen mit gedehnten Spalten:
% https://www.ctan.org/pkg/tabularx
%\usepackage{tabularx}
% vielleicht auch tabulary anschauen:
% https://www.ctan.org/pkg/tabulary
%\usepackage{tabulary}

% Pimpt enumerate auf (optionales Argument liefert Nummerierung):
\usepackage{enumerate}

% Kann descriptions auf die selbe Höhe bringen:
% https://www.ctan.org/pkg/enumitem
%\usepackage{enumitem}

% Liefert Hyperlinks (\hyperref, \url, \href}
\usepackage{hyperref}
%\hypersetup{%
%  colorlinks=false,
%  linkcolor=black,
%  urlcolor=blue,
%}

% To automatically typeset a cross-reference according to the type of thing referred to,
% simply refer to it using \cref{<label>}.
% https://www.ctan.org/pkg/cleveref
%\usepackage{cleveref}

% Farben (wie bei TikZ):
%\usepackage[dvipsnames]{xcolor}
%\definecolor{mygray}{gray}{0.8}

% Access to PostScript standard Symbol and Dingbats fonts
% https://www.ctan.org/pkg/pifont
%\usepackage[]{pifont}

% Ändert den Zeilenabstand:
%\usepackage[
%  % nur eine Möglichkeit auswählen:
%  singlespacing
%  %onehalfspacing
%  %doublespacing
%]{setspace}

% Schönere Tabellen
% dazu gibt's neue Kommandos:
% - \toprule[(Dicke)], \midrule[(Dicke)], \bottomrule[(Dicke)]
% - \addlinespace: Extrahöhe zwischen Zeilen
\usepackage{booktabs}

% TODOs:
%\usepackage[
%  ngerman,
%  textwidth=2cm,
%  textsize=tiny,
%  backgroundcolor=white,
%  linecolor=black,
%]{todonotes}

% Schöne numerische Zitierungen:
%\usepackage{cite}
%\usepackage[square, numbers]{natbib}

% Ermöglicht durch \begin{linenumbers} Zeilennummern anzuzeigen:
%\usepackage{lineno}

% Ermöglicht Zugriff auf die letzte Seite (z.B. \pageref{LastPage}):
\usepackage{lastpage}

% Logische Beweise:
%\usepackage{bussproofs}

% Unterstreichungen (\uline, \uuline, \sout: durchgestrichen, \uwave):
%\usepackage{ulem}

% Kann alle möglichen Maße ändern
% will man Querformat, dann:
%\usepackage[landscape]{geometry}

% bietet gestrichelte vert. Linien in Tabellen (':')
%\usepackage{arydshln}

% Quelltext schön setzen:
%\usepackage{listings}
% Quelltext noch schöner setzen:
% (verlangt "xelatex -shell-escape"!)
%\usepackage{minted}

% Algorithmen und Pseudocode:
%\usepackage{algorithm}
%\usepackage{algorithmic}
%\floatname{algorithm}{Algorithmus}
%\renewcommand{\algorithmicrequire}{\textbf{Eingabe:}}
%\renewcommand{\algorithmicensure}{\textbf{Ausgabe:}}

% Bilder einbinden:
%\usepackage{graphicx}
%\usepackage{subcaption}

% Verbessert den Satz von Abbildungsüberschriften:
%\usepackage{caption}

% um in Tabellen eine Zelle über mehrere Zeilen laufen zu lassen:
%\usepackage{multirow}

%\usepackage{float}

% SI-Einheiten mittels \si{}:
%\usepackage[mode=text]{siunitx}
%\sisetup{%
%  output-decimal-marker={,},
%  per-mode=fraction,
%  exponent-product=\cdot,
%}
%\DeclareSIUnit\cal{cal}
%\DeclareSIUnit\diopter{dpt}
%\DeclareSIUnit\fahrenheit{F}
%\DeclareSIUnit\molar{\textsc{m}}
%\DeclareSIUnit\pH{pH}
%\DeclareSIUnit\gewprozent{Gew\%}
%\DeclareSIUnit\poise{P}

% nette Brüche mittels \sfrac{}{}:
%\usepackage{xfrac}

% Coole Zeichnungen:
%\usepackage{tikz}
%\usetikzlibrary{%
  %backgrounds,
  %mindmap,
  %shapes.geometric,
  %shapes.symbols,
  %shapes.misc,
  %shapes.multipart,
  %positioning,
  %fit,
  %calc,
  %arrows,
  %automata,
  %trees,
  %decorations.pathreplacing,
  %circuits.ee.IEC,
  %intersections,
  %through,
%}

%\usepackage{pgfplots}
%\pgfplotsset{compat=1.16}

% eigens gebaute Lochmarken:
%\usepackage{eso-pic}
%\AddToShipoutPicture*{
  %\put(\LenToUnit{0mm},\LenToUnit{228.5mm})
    %{\rule{\LenToUnit{20pt}}{\LenToUnit{0.5pt}}}
  %\put(\LenToUnit{0mm},\LenToUnit{68.5mm})
    %{\rule{\LenToUnit{20pt}}{\LenToUnit{0.5pt}}}
%}

%\usepackage{titlesec}
%\titleformat*{\paragraph}{\itshape\mdseries} % chktex 6
% \titleformat{\section}
%   {\sffamily}{\thesection}{1em}{}

% ein Eintrag in einer description-Liste wird in ganz normaler Schrift angezeigt (kein
% sans-serif, kein fett):
%\renewcommand{\descriptionlabel}[1]{\hspace{\labelsep}#1}

%\usepackage{abstract}
%\addto\captionsngerman{\renewcommand{\abstractname}{Abstract}}

%\usepackage{epigraph}
%\setlength{\epigraphwidth}{0.42\textwidth}

% Definitionen und Sätze:
%\usepackage[]{amsthm}
% Definitionen, Probleme werden alle mit einem Zeilenumbruch gesetzt, z.B.:
%   Definition 1 (Graph)
%   Ein Graph G = (V, E) ist …
%\newtheoremstyle{bonny}% schottisch für „ansehnlich“
%  {9pt}% measure of space to leave above the theorem. E.g.: 3pt
%  {6pt}% measure of space to leave below the theorem. E.g.: 3pt
%  {}% name of font to use in the body of the theorem
%  {}% measure of space to indent
%  {\bfseries}% name of head font
%  {\smallskip}% punctuation between head and body
%  {\newline}% space after theorem head; " " = normal interword space
%  {}% Manually specify head

%\theoremstyle{bonny}
%\newtheorem{definition}{Definition}
%\newtheorem{problem}{Problem}

% Beispiele, Sätze, Theoreme, Lemmata werden alle *ohne* Zeilenumbruch gesetzt, z.B.:
%   Satz 1  Für alle Graphen gilt …
%\newtheoremstyle{sweet}%
%  {9pt}% measure of space to leave above the theorem. E.g.: 3pt
%  {6pt}% measure of space to leave below the theorem. E.g.: 3pt
%  {}% name of font to use in the body of the theorem
%  {}% measure of space to indent
%  {\bfseries}% name of head font
%  {}% punctuation between head and body
%  {1em}% space after theorem head; " " = normal interword space
%  {}% Manually specify head

%\theoremstyle{sweet}
%\newtheorem{beispiel}{Beispiel}
%\newtheorem{satz}{Satz}
%\newtheorem{theorem}{Theorem}
%\newtheorem{lemma}{Lemma}
%\newtheorem{folgerung}{Folgerung}

%% coole Kopf- und Fußzeilen:
%\usepackage{fancyhdr}
%% Seitenstil ist natürlich fancy:
%\pagestyle{fancy}
%% alle Felder löschen:
%\fancyhf{}
%% Veranstaltung:
%%\fancyhead[L]{}
%% Seriennummer:
%%\fancyhead[C]{}
%% Name und Matrikelnummer:
%%\fancyhead[R]{}
%%\fancyfoot[L]{}
%\fancyfoot[C]{\thepage}
%%\fancyfoot[C]{\thepage\,/\,\pageref{LastPage}}
%% Linie oben/unten:
%\renewcommand{\headrulewidth}{0.0pt}
%\renewcommand{\footrulewidth}{0.0pt}

%\newcommand{\cmark}{\ding{51}}%
%\newcommand{\xmark}{\ding{55}}%
%\newcommand{\richtig}{\textcolor{ForestGreen}{\cmark}}
%\newcommand{\falsch}{\textcolor{BrickRed}{\xmark}}

%\newcommand{\unterschrift}[2][5cm]{%
%  \begin{tabular}{@{}p{#1}@{}}
%    #2 \\[2\normalbaselineskip]
%    \hrule \\[-12pt]
%    {\small Unterschrift} \\[2\normalbaselineskip]
%    \hrule \\[-12pt]
%    {\small Datum}
%  \end{tabular}
%}

%\newcommand{\BigO}{\mathcal{O}}

%\newenvironment{notiz}{
%  \color{Maroon}
%  \paragraph*{Notiz}
%}{
%  \color{black}
%}

%\newcommand{\p}[1]{\text{p#1}}
%\newcommand{\pKs}{\text{pK$_\text{S}$}}
%\newcommand{\pKb}{\text{pK$_\text{B}$}}
 einbinden.
% Wenn die Dokumentenklasse tufte-handout oder tufte-book ist, dann wird das hier
% automatisch geladen.

\makeatletter
\@ifclassloaded{tufte-handout}{%
  % Bei Tufte sind die Schriften schon eingestellt.
  \typeout{Current class is tufte-handout, fonts are already configured.}
}{%
  \usepackage{ifxetex}
  \ifxetex
    \def\xelatexmainfont{Minion Pro}
    \def\xelatexmainboldfont{Minion Pro Bold}
    \def\xelatexsansfont{Gill Sans Light}
    \def\xelatexsansboldfont{Gill Sans}
    \def\xelatextitlepagefont{Gill Sans}
    \def\xelatexmonofont{Fira Mono}
    \def\xelatexmathfont{Minion Pro}

    % Deutsche Sprache bei Silbentrennung und Datum:
    \usepackage{polyglossia}
    \setdefaultlanguage[babelshorthands=true]{german}

    % Um auch schöne Schriftarten auswählen zu können:
    \usepackage[MnSymbol]{mathspec}

    % Wir wollen, dass alle unsere Schriften für TeX und einander angepasst sind:
    \defaultfontfeatures{
      Ligatures=TeX,
      Mapping=TeX-text,
      Scale=MatchLowercase,
    }

    % Die Hauptschriftart:
    \setmainfont{\xelatexmainfont}[
      BoldFont={\xelatexmainboldfont},
    ]
    % Die Matheschriftart:
    \setmathfont(Digits,Latin,Greek)[
      Numbers={Lining, Proportional}
    ]{\xelatexmathfont}
    \setmathrm{\xelatexmathfont}
    % Die Schriftart für serifenlose Texte (z.B. Überschriften):
    \setallsansfonts[
      BoldFont={\xelatexsansboldfont},
    ]{\xelatexsansfont}
    % Und die Schriftart für nichtproportionale Texte:
    \setallmonofonts[]{\xelatexmonofont}
    \newfontface\titlepagefont{\xelatextitlepagefont}
  \else
    \usepackage[utf8]{inputenc}

    % Deutsche Sprache bei Silbentrennung und Datum:
    \usepackage[ngerman]{babel}
  \fi
}
\makeatother

% _das_ Mathepaket schlechthin:
%\usepackage[
%  %% Nummerierung von Gleichungen links:
%  leqno,
%  %% Ausgabe von Gleichungen linksbündig:
%  fleqn,
%]{mathtools}
% und dazu noch ein paar Mathesymbole und so:
% (muss vor mathspec geladen werden)
%\usepackage{amsmath, amssymb}

%\usepackage{microtype}

% https://ctan.org/pkg/parskip
%\usepackage[
%%skip=0.5baselineskip plus 2pt,  % Der Abstand zwischen zwei Absätzen.
%%indent=0pt, % Die Einrückung der ersten Zeile eines Absatzes.
%%parfill=30pt, % Der Abstand des Endes der letzten Zeile eines Absatzes und dem rechten Seitenrand.
%]{parskip}

% chemische Formeln
% https://ctan.org/pkg/mhchem
%\usepackage{mhchem}


% St. Mary Road, liefert Symbole für theoretische Informatik (z.B. \lightning):
% https://www.ctan.org/pkg/stmaryrd
%\usepackage{stmaryrd}

% nutze die volle Seite als Satzspiegel:
% https://www.ctan.org/pkg/fullpage
%\usepackage[
  % Randbreite sei 1.5cm (sonst ist sie 1in):
  %cm,
  % Kopf- und Fußzeile werden miteinbezogen:
  %headings
%]{fullpage}

% verbesserte Tabellen
% https://www.ctan.org/pkg/array
% bietet u.a. die Spaltenmöglichkeit 'm{width}' = zentrierte Spalte mit fester Breite
%\usepackage{array}

% kann komplexe Linien in Tabellen ziehen:
% https://www.ctan.org/pkg/hhline
%\usepackage{hhline}

% mehrseitige Tabellen:
% https://www.ctan.org/pkg/longtable
%\usepackage{longtable}

% Tabellen mit gedehnten Spalten:
% https://www.ctan.org/pkg/tabularx
%\usepackage{tabularx}
% vielleicht auch tabulary anschauen:
% https://www.ctan.org/pkg/tabulary
%\usepackage{tabulary}

% Pimpt enumerate auf (optionales Argument liefert Nummerierung):
\usepackage{enumerate}

% Kann descriptions auf die selbe Höhe bringen:
% https://www.ctan.org/pkg/enumitem
%\usepackage{enumitem}

% Liefert Hyperlinks (\hyperref, \url, \href}
\usepackage{hyperref}
%\hypersetup{%
%  colorlinks=false,
%  linkcolor=black,
%  urlcolor=blue,
%}

% To automatically typeset a cross-reference according to the type of thing referred to,
% simply refer to it using \cref{<label>}.
% https://www.ctan.org/pkg/cleveref
%\usepackage{cleveref}

% Farben (wie bei TikZ):
%\usepackage[dvipsnames]{xcolor}
%\definecolor{mygray}{gray}{0.8}

% Access to PostScript standard Symbol and Dingbats fonts
% https://www.ctan.org/pkg/pifont
%\usepackage[]{pifont}

% Ändert den Zeilenabstand:
%\usepackage[
%  % nur eine Möglichkeit auswählen:
%  singlespacing
%  %onehalfspacing
%  %doublespacing
%]{setspace}

% Schönere Tabellen
% dazu gibt's neue Kommandos:
% - \toprule[(Dicke)], \midrule[(Dicke)], \bottomrule[(Dicke)]
% - \addlinespace: Extrahöhe zwischen Zeilen
\usepackage{booktabs}

% TODOs:
%\usepackage[
%  ngerman,
%  textwidth=2cm,
%  textsize=tiny,
%  backgroundcolor=white,
%  linecolor=black,
%]{todonotes}

% Schöne numerische Zitierungen:
%\usepackage{cite}
%\usepackage[square, numbers]{natbib}

% Ermöglicht durch \begin{linenumbers} Zeilennummern anzuzeigen:
%\usepackage{lineno}

% Ermöglicht Zugriff auf die letzte Seite (z.B. \pageref{LastPage}):
\usepackage{lastpage}

% Logische Beweise:
%\usepackage{bussproofs}

% Unterstreichungen (\uline, \uuline, \sout: durchgestrichen, \uwave):
%\usepackage{ulem}

% Kann alle möglichen Maße ändern
% will man Querformat, dann:
%\usepackage[landscape]{geometry}

% bietet gestrichelte vert. Linien in Tabellen (':')
%\usepackage{arydshln}

% Quelltext schön setzen:
%\usepackage{listings}
% Quelltext noch schöner setzen:
% (verlangt "xelatex -shell-escape"!)
%\usepackage{minted}

% Algorithmen und Pseudocode:
%\usepackage{algorithm}
%\usepackage{algorithmic}
%\floatname{algorithm}{Algorithmus}
%\renewcommand{\algorithmicrequire}{\textbf{Eingabe:}}
%\renewcommand{\algorithmicensure}{\textbf{Ausgabe:}}

% Bilder einbinden:
%\usepackage{graphicx}
%\usepackage{subcaption}

% Verbessert den Satz von Abbildungsüberschriften:
%\usepackage{caption}

% um in Tabellen eine Zelle über mehrere Zeilen laufen zu lassen:
%\usepackage{multirow}

%\usepackage{float}

% SI-Einheiten mittels \si{}:
%\usepackage[mode=text]{siunitx}
%\sisetup{%
%  output-decimal-marker={,},
%  per-mode=fraction,
%  exponent-product=\cdot,
%}
%\DeclareSIUnit\cal{cal}
%\DeclareSIUnit\diopter{dpt}
%\DeclareSIUnit\fahrenheit{F}
%\DeclareSIUnit\molar{\textsc{m}}
%\DeclareSIUnit\pH{pH}
%\DeclareSIUnit\gewprozent{Gew\%}
%\DeclareSIUnit\poise{P}

% nette Brüche mittels \sfrac{}{}:
%\usepackage{xfrac}

% Coole Zeichnungen:
%\usepackage{tikz}
%\usetikzlibrary{%
  %backgrounds,
  %mindmap,
  %shapes.geometric,
  %shapes.symbols,
  %shapes.misc,
  %shapes.multipart,
  %positioning,
  %fit,
  %calc,
  %arrows,
  %automata,
  %trees,
  %decorations.pathreplacing,
  %circuits.ee.IEC,
  %intersections,
  %through,
%}

%\usepackage{pgfplots}
%\pgfplotsset{compat=1.16}

% eigens gebaute Lochmarken:
%\usepackage{eso-pic}
%\AddToShipoutPicture*{
  %\put(\LenToUnit{0mm},\LenToUnit{228.5mm})
    %{\rule{\LenToUnit{20pt}}{\LenToUnit{0.5pt}}}
  %\put(\LenToUnit{0mm},\LenToUnit{68.5mm})
    %{\rule{\LenToUnit{20pt}}{\LenToUnit{0.5pt}}}
%}

%\usepackage{titlesec}
%\titleformat*{\paragraph}{\itshape\mdseries} % chktex 6
% \titleformat{\section}
%   {\sffamily}{\thesection}{1em}{}

% ein Eintrag in einer description-Liste wird in ganz normaler Schrift angezeigt (kein
% sans-serif, kein fett):
%\renewcommand{\descriptionlabel}[1]{\hspace{\labelsep}#1}

%\usepackage{abstract}
%\addto\captionsngerman{\renewcommand{\abstractname}{Abstract}}

%\usepackage{epigraph}
%\setlength{\epigraphwidth}{0.42\textwidth}

% Definitionen und Sätze:
%\usepackage[]{amsthm}
% Definitionen, Probleme werden alle mit einem Zeilenumbruch gesetzt, z.B.:
%   Definition 1 (Graph)
%   Ein Graph G = (V, E) ist …
%\newtheoremstyle{bonny}% schottisch für „ansehnlich“
%  {9pt}% measure of space to leave above the theorem. E.g.: 3pt
%  {6pt}% measure of space to leave below the theorem. E.g.: 3pt
%  {}% name of font to use in the body of the theorem
%  {}% measure of space to indent
%  {\bfseries}% name of head font
%  {\smallskip}% punctuation between head and body
%  {\newline}% space after theorem head; " " = normal interword space
%  {}% Manually specify head

%\theoremstyle{bonny}
%\newtheorem{definition}{Definition}
%\newtheorem{problem}{Problem}

% Beispiele, Sätze, Theoreme, Lemmata werden alle *ohne* Zeilenumbruch gesetzt, z.B.:
%   Satz 1  Für alle Graphen gilt …
%\newtheoremstyle{sweet}%
%  {9pt}% measure of space to leave above the theorem. E.g.: 3pt
%  {6pt}% measure of space to leave below the theorem. E.g.: 3pt
%  {}% name of font to use in the body of the theorem
%  {}% measure of space to indent
%  {\bfseries}% name of head font
%  {}% punctuation between head and body
%  {1em}% space after theorem head; " " = normal interword space
%  {}% Manually specify head

%\theoremstyle{sweet}
%\newtheorem{beispiel}{Beispiel}
%\newtheorem{satz}{Satz}
%\newtheorem{theorem}{Theorem}
%\newtheorem{lemma}{Lemma}
%\newtheorem{folgerung}{Folgerung}

%% coole Kopf- und Fußzeilen:
%\usepackage{fancyhdr}
%% Seitenstil ist natürlich fancy:
%\pagestyle{fancy}
%% alle Felder löschen:
%\fancyhf{}
%% Veranstaltung:
%%\fancyhead[L]{}
%% Seriennummer:
%%\fancyhead[C]{}
%% Name und Matrikelnummer:
%%\fancyhead[R]{}
%%\fancyfoot[L]{}
%\fancyfoot[C]{\thepage}
%%\fancyfoot[C]{\thepage\,/\,\pageref{LastPage}}
%% Linie oben/unten:
%\renewcommand{\headrulewidth}{0.0pt}
%\renewcommand{\footrulewidth}{0.0pt}

%\newcommand{\cmark}{\ding{51}}%
%\newcommand{\xmark}{\ding{55}}%
%\newcommand{\richtig}{\textcolor{ForestGreen}{\cmark}}
%\newcommand{\falsch}{\textcolor{BrickRed}{\xmark}}

%\newcommand{\unterschrift}[2][5cm]{%
%  \begin{tabular}{@{}p{#1}@{}}
%    #2 \\[2\normalbaselineskip]
%    \hrule \\[-12pt]
%    {\small Unterschrift} \\[2\normalbaselineskip]
%    \hrule \\[-12pt]
%    {\small Datum}
%  \end{tabular}
%}

%\newcommand{\BigO}{\mathcal{O}}

%\newenvironment{notiz}{
%  \color{Maroon}
%  \paragraph*{Notiz}
%}{
%  \color{black}
%}

%\newcommand{\p}[1]{\text{p#1}}
%\newcommand{\pKs}{\text{pK$_\text{S}$}}
%\newcommand{\pKb}{\text{pK$_\text{B}$}}
 einbinden.
% Wenn die Dokumentenklasse tufte-handout oder tufte-book ist, dann wird das hier
% automatisch geladen.

\makeatletter
\@ifclassloaded{tufte-handout}{%
  % Bei Tufte sind die Schriften schon eingestellt.
  \typeout{Current class is tufte-handout, fonts are already configured.}
}{%
  \usepackage{ifxetex}
  \ifxetex
    \def\xelatexmainfont{Minion Pro}
    \def\xelatexmainboldfont{Minion Pro Bold}
    \def\xelatexsansfont{Gill Sans Light}
    \def\xelatexsansboldfont{Gill Sans}
    \def\xelatextitlepagefont{Gill Sans}
    \def\xelatexmonofont{Fira Mono}
    \def\xelatexmathfont{Minion Pro}

    % Deutsche Sprache bei Silbentrennung und Datum:
    \usepackage{polyglossia}
    \setdefaultlanguage[babelshorthands=true]{german}

    % Um auch schöne Schriftarten auswählen zu können:
    \usepackage[MnSymbol]{mathspec}

    % Wir wollen, dass alle unsere Schriften für TeX und einander angepasst sind:
    \defaultfontfeatures{
      Ligatures=TeX,
      Mapping=TeX-text,
      Scale=MatchLowercase,
    }

    % Die Hauptschriftart:
    \setmainfont{\xelatexmainfont}[
      BoldFont={\xelatexmainboldfont},
    ]
    % Die Matheschriftart:
    \setmathfont(Digits,Latin,Greek)[
      Numbers={Lining, Proportional}
    ]{\xelatexmathfont}
    \setmathrm{\xelatexmathfont}
    % Die Schriftart für serifenlose Texte (z.B. Überschriften):
    \setallsansfonts[
      BoldFont={\xelatexsansboldfont},
    ]{\xelatexsansfont}
    % Und die Schriftart für nichtproportionale Texte:
    \setallmonofonts[]{\xelatexmonofont}
    \newfontface\titlepagefont{\xelatextitlepagefont}
  \else
    \usepackage[utf8]{inputenc}

    % Deutsche Sprache bei Silbentrennung und Datum:
    \usepackage[ngerman]{babel}
  \fi
}
\makeatother

% _das_ Mathepaket schlechthin:
%\usepackage[
%  %% Nummerierung von Gleichungen links:
%  leqno,
%  %% Ausgabe von Gleichungen linksbündig:
%  fleqn,
%]{mathtools}
% und dazu noch ein paar Mathesymbole und so:
% (muss vor mathspec geladen werden)
%\usepackage{amsmath, amssymb}

%\usepackage{microtype}

% https://ctan.org/pkg/parskip
%\usepackage[
%%skip=0.5baselineskip plus 2pt,  % Der Abstand zwischen zwei Absätzen.
%%indent=0pt, % Die Einrückung der ersten Zeile eines Absatzes.
%%parfill=30pt, % Der Abstand des Endes der letzten Zeile eines Absatzes und dem rechten Seitenrand.
%]{parskip}

% chemische Formeln
% https://ctan.org/pkg/mhchem
%\usepackage{mhchem}


% St. Mary Road, liefert Symbole für theoretische Informatik (z.B. \lightning):
% https://www.ctan.org/pkg/stmaryrd
%\usepackage{stmaryrd}

% nutze die volle Seite als Satzspiegel:
% https://www.ctan.org/pkg/fullpage
%\usepackage[
  % Randbreite sei 1.5cm (sonst ist sie 1in):
  %cm,
  % Kopf- und Fußzeile werden miteinbezogen:
  %headings
%]{fullpage}

% verbesserte Tabellen
% https://www.ctan.org/pkg/array
% bietet u.a. die Spaltenmöglichkeit 'm{width}' = zentrierte Spalte mit fester Breite
%\usepackage{array}

% kann komplexe Linien in Tabellen ziehen:
% https://www.ctan.org/pkg/hhline
%\usepackage{hhline}

% mehrseitige Tabellen:
% https://www.ctan.org/pkg/longtable
%\usepackage{longtable}

% Tabellen mit gedehnten Spalten:
% https://www.ctan.org/pkg/tabularx
%\usepackage{tabularx}
% vielleicht auch tabulary anschauen:
% https://www.ctan.org/pkg/tabulary
%\usepackage{tabulary}

% Pimpt enumerate auf (optionales Argument liefert Nummerierung):
\usepackage{enumerate}

% Kann descriptions auf die selbe Höhe bringen:
% https://www.ctan.org/pkg/enumitem
%\usepackage{enumitem}

% Liefert Hyperlinks (\hyperref, \url, \href}
\usepackage{hyperref}
%\hypersetup{%
%  colorlinks=false,
%  linkcolor=black,
%  urlcolor=blue,
%}

% To automatically typeset a cross-reference according to the type of thing referred to,
% simply refer to it using \cref{<label>}.
% https://www.ctan.org/pkg/cleveref
%\usepackage{cleveref}

% Farben (wie bei TikZ):
%\usepackage[dvipsnames]{xcolor}
%\definecolor{mygray}{gray}{0.8}

% Access to PostScript standard Symbol and Dingbats fonts
% https://www.ctan.org/pkg/pifont
%\usepackage[]{pifont}

% Ändert den Zeilenabstand:
%\usepackage[
%  % nur eine Möglichkeit auswählen:
%  singlespacing
%  %onehalfspacing
%  %doublespacing
%]{setspace}

% Schönere Tabellen
% dazu gibt's neue Kommandos:
% - \toprule[(Dicke)], \midrule[(Dicke)], \bottomrule[(Dicke)]
% - \addlinespace: Extrahöhe zwischen Zeilen
\usepackage{booktabs}

% TODOs:
%\usepackage[
%  ngerman,
%  textwidth=2cm,
%  textsize=tiny,
%  backgroundcolor=white,
%  linecolor=black,
%]{todonotes}

% Schöne numerische Zitierungen:
%\usepackage{cite}
%\usepackage[square, numbers]{natbib}

% Ermöglicht durch \begin{linenumbers} Zeilennummern anzuzeigen:
%\usepackage{lineno}

% Ermöglicht Zugriff auf die letzte Seite (z.B. \pageref{LastPage}):
\usepackage{lastpage}

% Logische Beweise:
%\usepackage{bussproofs}

% Unterstreichungen (\uline, \uuline, \sout: durchgestrichen, \uwave):
%\usepackage{ulem}

% Kann alle möglichen Maße ändern
% will man Querformat, dann:
%\usepackage[landscape]{geometry}

% bietet gestrichelte vert. Linien in Tabellen (':')
%\usepackage{arydshln}

% Quelltext schön setzen:
%\usepackage{listings}
% Quelltext noch schöner setzen:
% (verlangt "xelatex -shell-escape"!)
%\usepackage{minted}

% Algorithmen und Pseudocode:
%\usepackage{algorithm}
%\usepackage{algorithmic}
%\floatname{algorithm}{Algorithmus}
%\renewcommand{\algorithmicrequire}{\textbf{Eingabe:}}
%\renewcommand{\algorithmicensure}{\textbf{Ausgabe:}}

% Bilder einbinden:
%\usepackage{graphicx}
%\usepackage{subcaption}

% Verbessert den Satz von Abbildungsüberschriften:
%\usepackage{caption}

% um in Tabellen eine Zelle über mehrere Zeilen laufen zu lassen:
%\usepackage{multirow}

%\usepackage{float}

% SI-Einheiten mittels \si{}:
%\usepackage[mode=text]{siunitx}
%\sisetup{%
%  output-decimal-marker={,},
%  per-mode=fraction,
%  exponent-product=\cdot,
%}
%\DeclareSIUnit\cal{cal}
%\DeclareSIUnit\diopter{dpt}
%\DeclareSIUnit\fahrenheit{F}
%\DeclareSIUnit\molar{\textsc{m}}
%\DeclareSIUnit\pH{pH}
%\DeclareSIUnit\gewprozent{Gew\%}
%\DeclareSIUnit\poise{P}

% nette Brüche mittels \sfrac{}{}:
%\usepackage{xfrac}

% Coole Zeichnungen:
%\usepackage{tikz}
%\usetikzlibrary{%
  %backgrounds,
  %mindmap,
  %shapes.geometric,
  %shapes.symbols,
  %shapes.misc,
  %shapes.multipart,
  %positioning,
  %fit,
  %calc,
  %arrows,
  %automata,
  %trees,
  %decorations.pathreplacing,
  %circuits.ee.IEC,
  %intersections,
  %through,
%}

%\usepackage{pgfplots}
%\pgfplotsset{compat=1.16}

% eigens gebaute Lochmarken:
%\usepackage{eso-pic}
%\AddToShipoutPicture*{
  %\put(\LenToUnit{0mm},\LenToUnit{228.5mm})
    %{\rule{\LenToUnit{20pt}}{\LenToUnit{0.5pt}}}
  %\put(\LenToUnit{0mm},\LenToUnit{68.5mm})
    %{\rule{\LenToUnit{20pt}}{\LenToUnit{0.5pt}}}
%}

%\usepackage{titlesec}
%\titleformat*{\paragraph}{\itshape\mdseries} % chktex 6
% \titleformat{\section}
%   {\sffamily}{\thesection}{1em}{}

% ein Eintrag in einer description-Liste wird in ganz normaler Schrift angezeigt (kein
% sans-serif, kein fett):
%\renewcommand{\descriptionlabel}[1]{\hspace{\labelsep}#1}

%\usepackage{abstract}
%\addto\captionsngerman{\renewcommand{\abstractname}{Abstract}}

%\usepackage{epigraph}
%\setlength{\epigraphwidth}{0.42\textwidth}

% Definitionen und Sätze:
%\usepackage[]{amsthm}
% Definitionen, Probleme werden alle mit einem Zeilenumbruch gesetzt, z.B.:
%   Definition 1 (Graph)
%   Ein Graph G = (V, E) ist …
%\newtheoremstyle{bonny}% schottisch für „ansehnlich“
%  {9pt}% measure of space to leave above the theorem. E.g.: 3pt
%  {6pt}% measure of space to leave below the theorem. E.g.: 3pt
%  {}% name of font to use in the body of the theorem
%  {}% measure of space to indent
%  {\bfseries}% name of head font
%  {\smallskip}% punctuation between head and body
%  {\newline}% space after theorem head; " " = normal interword space
%  {}% Manually specify head

%\theoremstyle{bonny}
%\newtheorem{definition}{Definition}
%\newtheorem{problem}{Problem}

% Beispiele, Sätze, Theoreme, Lemmata werden alle *ohne* Zeilenumbruch gesetzt, z.B.:
%   Satz 1  Für alle Graphen gilt …
%\newtheoremstyle{sweet}%
%  {9pt}% measure of space to leave above the theorem. E.g.: 3pt
%  {6pt}% measure of space to leave below the theorem. E.g.: 3pt
%  {}% name of font to use in the body of the theorem
%  {}% measure of space to indent
%  {\bfseries}% name of head font
%  {}% punctuation between head and body
%  {1em}% space after theorem head; " " = normal interword space
%  {}% Manually specify head

%\theoremstyle{sweet}
%\newtheorem{beispiel}{Beispiel}
%\newtheorem{satz}{Satz}
%\newtheorem{theorem}{Theorem}
%\newtheorem{lemma}{Lemma}
%\newtheorem{folgerung}{Folgerung}

%% coole Kopf- und Fußzeilen:
%\usepackage{fancyhdr}
%% Seitenstil ist natürlich fancy:
%\pagestyle{fancy}
%% alle Felder löschen:
%\fancyhf{}
%% Veranstaltung:
%%\fancyhead[L]{}
%% Seriennummer:
%%\fancyhead[C]{}
%% Name und Matrikelnummer:
%%\fancyhead[R]{}
%%\fancyfoot[L]{}
%\fancyfoot[C]{\thepage}
%%\fancyfoot[C]{\thepage\,/\,\pageref{LastPage}}
%% Linie oben/unten:
%\renewcommand{\headrulewidth}{0.0pt}
%\renewcommand{\footrulewidth}{0.0pt}

%\newcommand{\cmark}{\ding{51}}%
%\newcommand{\xmark}{\ding{55}}%
%\newcommand{\richtig}{\textcolor{ForestGreen}{\cmark}}
%\newcommand{\falsch}{\textcolor{BrickRed}{\xmark}}

%\newcommand{\unterschrift}[2][5cm]{%
%  \begin{tabular}{@{}p{#1}@{}}
%    #2 \\[2\normalbaselineskip]
%    \hrule \\[-12pt]
%    {\small Unterschrift} \\[2\normalbaselineskip]
%    \hrule \\[-12pt]
%    {\small Datum}
%  \end{tabular}
%}

%\newcommand{\BigO}{\mathcal{O}}

%\newenvironment{notiz}{
%  \color{Maroon}
%  \paragraph*{Notiz}
%}{
%  \color{black}
%}

%\newcommand{\p}[1]{\text{p#1}}
%\newcommand{\pKs}{\text{pK$_\text{S}$}}
%\newcommand{\pKb}{\text{pK$_\text{B}$}}
